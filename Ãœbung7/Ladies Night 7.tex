\documentclass[12pt,a4paper,oneside,ngerman]{article} 
\usepackage[left=3cm,right=3cm,top=2.5cm]{geometry} % Groesse der Seitenraender definieren
\usepackage[utf8]{inputenc} % utf8 encoding
\usepackage{hyperref}
\usepackage{ngerman}[babel]
\usepackage{graphicx}
\usepackage{tikz} % Automaten, Graphen, ... zeichnen
\usepackage{tikz-qtree} % Paket fuer Tikz Graphen-Baeume
\usetikzlibrary{arrows,shapes,automata} % Bestimmte tikz-Befehle benutzen
\usepackage{amsmath,amssymb} % Mathe-Formeln und -Ausdruecke
\usepackage{listings} % Code-Ausschnitte einbinden
\usepackage{xcolor} % Eigene Farben definieren
\usepackage{colortbl} % Farben verwenden in Tabellen
\usepackage{wrapfig} % Bilder von Text umfliessen lassen
\usepackage{multicol} % Mehrspaltigen Text schreiben
\usepackage{stmaryrd} % Fuer Symbole wie zu Beispiel Widerspruchspfeil
\usepackage{caption}
\usepackage{totpages}

\usepackage{circuitikz}
\usepackage{amsmath}
\usepackage{graphicx}
\usepackage{subfigure}
\usepackage{tikz}
\usepackage{float}

\lstset{language=Java,
basicstyle=\small \ttfamily,
keywordstyle=\color{javapurple}\bfseries,
stringstyle=\color{javared},
commentstyle=\color{javagreen},
morecomment=[s][\color{javadocblue}]{/**}{*/},
numbers=left,
numberstyle=\tiny\color{black},
stepnumber=1,
numbersep=10pt,
tabsize=1,
showspaces=false,
showstringspaces=false,
breaklines=true}
\lstset{literate=%
  {Ö}{{\"O}}1
  {Ä}{{\"A}}1
  {Ü}{{\"U}}1
  {ß}{{\ss}}1
  {ü}{{\"u}}1
  {ä}{{\"a}}1
  {ö}{{\"o}}1
}

% Beliebige RGB Farben definieren:
\definecolor{gold}{rgb}{0.83, 0.69, 0.15}
\definecolor{magenta}{rgb}{0.79, 0.08, 0.48}
\definecolor{javared}{rgb}{0.6,0,0} % for strings
\definecolor{javagreen}{rgb}{0.25,0.5,0.35} % comments
\definecolor{javapurple}{rgb}{0.5,0,0.35} % keywords
\definecolor{javadocblue}{rgb}{0.25,0.35,0.75} % javadoc

% Beliebige RGB Farben definieren:
\definecolor{gold}{rgb}{0.83, 0.69, 0.15}
\definecolor{magenta}{rgb}{0.79, 0.08, 0.48}

% Titel in Kopfzeilen
\usepackage{fancyhdr}
\pagestyle{fancy}
\fancyhf{}
\setlength{\headheight}{20pt}

% Seitenumbrueche werden nicht mehr eingerueckt
\setlength{\parindent}{0em}


% % % % % % % % % % % % % % % % % % % % % % % % % % % % % % 
%Variablen
% % % % % % % % % % % % % % % % % % % % % % % % % % % % % % 
\newcommand{\fach}{Objektorientierte Modellierung und Programmierung}
\newcommand{\dokumentenTitel}{Abgabe Uebungsblatt Nr.07}
\newcommand{\Abgabe}{09.06.2020, 12:00 Uhr}
\newcommand{\memberOne}{Marius Birk}
\newcommand{\memberTwo}{Pieter Vogt}


\newcommand{\tutor}{ Florian Brandt }
% % % % % % % % % % % % % % % % % % % % % % % % % % % % % 

% Kopfzeile auf jeder Seite:
\fancyhead[R]{\dokumentenTitel} % Dokument-Titel
\fancyhead[C]{}
\fancyhead[L]{\memberOne, \memberTwo} % Autorennamen
\renewcommand{\headrulewidth}{0.4pt} %obere Trennlinie

% Fußzeile auf jeder Seite:
\fancyfoot[C]{Seite \thepage \ von \ref{TotPages}} %Seitennummer
\renewcommand{\footrulewidth}{0.4pt} %untere Trennlinie

% Nun beginnt das eigentliche Dokument
\begin{document}
	\thispagestyle{plain} % Keine Kopfzeile auf erster Seite, aber Seitenzahl wird angezeigt
	
	\begin{multicols}{2} % Beginnt zweispaltigen Text fuer Header auf erster Seite
		\hspace{-1cm} % Linken Header-Teil 1cm nach links schieben.
		% Tabelle fuer linke Seite vom Header der ersten Seite
		\begin{tabular}{ll} % Mit l werden die Eintraege linksbuendig
			Autoren: & \memberOne \\ % Zwischen jeder Spalte ein & einfuegen
			& \memberTwo \\
% beendet eine Tabellenzeile 
			Tutor: & \tutor \\  
		\end{tabular}
		
		\columnbreak % Nun beginnt die rechte Seite des Headers
		\hspace{-1cm} % Rechten Header-Teil 1cm nach links schieben.
		% Tabelle fuer rechte Seite vom Header der ersten Seite
		\begin{tabular}{ll} % p{1cm} bewirkt, dass die rechte Spalte 6cm breit ist.
			Abgabe: & \Abgabe \\ % Zwischen jeder Spalte ein & einfuege
			Smileys: &  
			%Mit diesem Befehl wird die Zeilenhoehe der folgenden Tabelle um 20% erhoeht.   
			\renewcommand{\arraystretch}{1.2} 
			% Nun kommt eine innere Tabelle in der aeusseren Tabelle, mit der eine Punktetabelle fuer den Tutor erstellt wird:  
			\begin{tabular}{|p{0.8cm}|p{0.8cm}|p{0.8cm}|p{0.8cm}|p{0.8cm}|}
				\hline A1 & A2 & A3 &$\sum\limits^{ }$ \\ \hline
				& & & \\ \hline    
			\end{tabular} \\
		\end{tabular}
		
	\end{multicols} % Beendet zweispaltigen Text
	
	\begin{center}
		\Large{\fach} \\
		\LARGE{\dokumentenTitel} \\
		\small
		$($Alle allgemeinen Definitionen aus der Vorlesung haben in diesem Dokument bestand, es sei den sie erhalten eine explizit andere Definition.$)$
	\end{center}


%~~~~~DOKUMENT ANFANG~~~~~%

%~~~~~Aufgabe 1
\section*{Aufgabe 1}
\begin{lstlisting}
   public class LambdaTest {

   public static void main(String[] args) {
      //@SuppressWarnings("unchecked")
      //Function<Double, Double> chain = makeChain(new Function[]{inverse, id, timesTen, divideByPi});
      //AUFGABE 1 B

      // i) map x to itself
      Function<Double> id = (x) -> x;
      System.out.println(id.calculate(5.0));

      //ii) map x to inverse self:
      Function<Double> inverse = (x) -> x * -1;
      System.out.println(inverse.calculate(5.0));

      //iii) map x to x*10
      Function<Double> timesTen = (x) -> x * 10;
      System.out.println(timesTen.calculate(5.0));

      //iv) map x to x/pi
      Function<Double> divideByPi = (x) -> x / Math.PI;
      System.out.println(divideByPi.calculate(5.0));


      //AUFGABE 1 C
      Function round = (x) -> Math.round(x.doubleValue());
      System.out.println(round.calculate(5.5421235223));
      System.out.println(round.calculate(5.5421235223).getClass().toString());

      //AUFGABE 1 D
      Function testChain = (x) -> x.doubleValue();
      Number a = inverse.calculate(id.calculate(timesTen.calculate(divideByPi.calculate(5.0))));

   }
/*
   public Function makeChain(final Function[] funs, Number n) {
      Function c;
      c =(x) -> funs[];

      return c;
   }
 */
}

\end{lstlisting}
\section*{Aufgabe 2}
\begin{lstlisting}
   import java.util.stream.Stream;
public class StreamTest {
    public static void main(final String[] args){
        final Stream<Integer> naturals =  Stream.iterate(1, x->x+1);
        final Stream<Integer> integers = Stream.iterate(0, (Integer x) ->{
            if(x<=0){
                x=x*-1;
                x=x+1;
            }else{
                x=x*-1;
            }
            return x;
        });
        
        System.out.println("Naturals:" + filterAndSum(naturals));
        System.out.print("Integers:" + filterAndSum(integers));
    }

    public static int filterAndSum(Stream<Integer> stream) {
        Stream<Integer> result = stream.filter(x -> x % 2 == 0).limit(10);
        int result1 = result.mapToInt(Integer::intValue).sum();
        if (result != null) {
            return result1;
        } else {
            return 0;
        }
    }
}
\end{lstlisting}
\section*{Aufgabe 3}
\begin{lstlisting}
import java.io.*;
import java.util.ArrayList;
import java.util.List;

class Person implements Serializable {
	private String firstname;
	private String lastname;
	private String sortname;
	public Person() { }
	public Person(String firstname, String lastname) {
		this.firstname = firstname;
		this.lastname = lastname;
		updateSortname();
	}
	public String getFirstname() {
		return firstname;
	}
	public void setFirstname(String firstname) {
		this.firstname = firstname;
		updateSortname();
	}
	public String getLastname() {
		return lastname;
	}
	public void setLastname(String lastname) {
		this.lastname = lastname;
		updateSortname();
	}
	public String getSortname() {
		return sortname;
	}
	public void updateSortname() {
		sortname = lastname + firstname;
	}
	@Override
	public String toString() {
		return firstname + " " + lastname + " (" + sortname + ")";
	}
	public static List<Person> load(String filename) throws IOException {
		List<Person> persons = new ArrayList<>();
		DataInputStream f = new DataInputStream(new BufferedInputStream(new FileInputStream(filename)));
		try{
			while(f != null){
				try{
					persons.add(load(f));
				}catch (EOFException e){
					break;
				}
			}
		} catch( FileNotFoundException e){
			System.out.print("File not found!");
		}catch (IOException e){
			System.out.print("End of File");
		} catch (ClassNotFoundException e) {
			e.printStackTrace();
		}finally {
			f.close();
		}
		return persons;
	}
	public static Person load(DataInputStream in) throws IOException, ClassNotFoundException {
		Person person = null;
		ObjectInputStream o = new ObjectInputStream(in);
		if(in != null){
			person=(Person) o.readObject();
		}
		return person;
	}
	public static void save(String filename, List<Person> list) throws IOException {
		DataOutputStream f = new DataOutputStream(new BufferedOutputStream(new FileOutputStream(filename)));
		try{
			for(int i = 0; i<list.size();i++){
				save(f, list.get(i));
			}
		}catch(IOException e){
			System.out.print(e);
		}finally {
			f.close();
		}
	}
	public static void save(DataOutputStream out, Person person) throws IOException {
		ObjectOutputStream o = new ObjectOutputStream(out);
		o.writeObject(person);
	}
	public static List<Person> unserialize(String filename) throws IOException, ClassNotFoundException {
		List<Person> persons = new ArrayList<>();
		DataInputStream f = new DataInputStream(new BufferedInputStream(new FileInputStream(filename)));
		ObjectInputStream o = new ObjectInputStream(f);
		try{
			while(f != null){
				try{
					persons.add((Person)o.readObject());
				}catch (EOFException e){
					break;
				}
			}
		} catch( FileNotFoundException e){
			System.out.print("File not found!");
		}catch (IOException e){
			System.out.print(e);
		} catch (ClassNotFoundException e) {
			e.printStackTrace();
		}finally {
			f.close();
		}
		return persons;
	}
	public static void serialize(String filename, List<Person> persons) throws IOException {
		DataOutputStream f = new DataOutputStream(new BufferedOutputStream(new FileOutputStream(filename)));
		ObjectOutputStream o = new ObjectOutputStream(f);
		try{
			for(int i = 0; i<persons.size();i++){
				o.writeObject(persons.get(i));
			}
		}catch(IOException e){
			System.out.print("Error initialize Output123");
		}finally {
			f.close();
			o.close();
		}
	}
}
public class PersonTest {
	public static void main(String[] args) throws IOException, ClassNotFoundException {
		List<Person> persons = new ArrayList<>();
		persons.add(new Person("Willy", "Wonka"));
		persons.add(new Person("Charlie", "Bucket"));
		persons.add(new Person("Grandpa", "Joe"));
		System.out.println(persons);
		
		Person.save("persons.sav", persons);
		persons = Person.load("persons.sav");
		System.out.println(persons);
		Person.serialize("persons.ser", persons);
		persons = Person.unserialize("persons.ser");
		System.out.println(persons);
	}

}
\end{lstlisting}
\end{document}

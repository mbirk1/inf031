\documentclass[12pt,a4paper,oneside,ngerman]{article} 
\usepackage[left=3cm,right=3cm,top=2.5cm]{geometry} % Groesse der Seitenraender definieren
\usepackage[utf8]{inputenc} % utf8 encoding
\usepackage{hyperref}
\usepackage{ngerman}[babel]
\usepackage{graphicx}
\usepackage{tikz} % Automaten, Graphen, ... zeichnen
\usepackage{tikz-qtree} % Paket fuer Tikz Graphen-Baeume
\usetikzlibrary{arrows,shapes,automata} % Bestimmte tikz-Befehle benutzen
\usepackage{amsmath,amssymb} % Mathe-Formeln und -Ausdruecke
\usepackage{listings} % Code-Ausschnitte einbinden
\usepackage{xcolor} % Eigene Farben definieren
\usepackage{colortbl} % Farben verwenden in Tabellen
\usepackage{wrapfig} % Bilder von Text umfliessen lassen
\usepackage{multicol} % Mehrspaltigen Text schreiben
\usepackage{stmaryrd} % Fuer Symbole wie zu Beispiel Widerspruchspfeil
\usepackage{caption}
\usepackage{totpages}

\usepackage{circuitikz}
\usepackage{amsmath}
\usepackage{graphicx}
\usepackage{subfigure}
\usepackage{tikz}
\usepackage{float}

\lstset{language=Java,
basicstyle=\small \ttfamily,
keywordstyle=\color{javapurple}\bfseries,
stringstyle=\color{javared},
commentstyle=\color{javagreen},
morecomment=[s][\color{javadocblue}]{/**}{*/},
numbers=left,
numberstyle=\tiny\color{black},
stepnumber=1,
numbersep=10pt,
tabsize=1,
showspaces=false,
showstringspaces=false,
breaklines=true}
\lstset{literate=%
  {Ö}{{\"O}}1
  {Ä}{{\"A}}1
  {Ü}{{\"U}}1
  {ß}{{\ss}}1
  {ü}{{\"u}}1
  {ä}{{\"a}}1
  {ö}{{\"o}}1
}

% Beliebige RGB Farben definieren:
\definecolor{gold}{rgb}{0.83, 0.69, 0.15}
\definecolor{magenta}{rgb}{0.79, 0.08, 0.48}
\definecolor{javared}{rgb}{0.6,0,0} % for strings
\definecolor{javagreen}{rgb}{0.25,0.5,0.35} % comments
\definecolor{javapurple}{rgb}{0.5,0,0.35} % keywords
\definecolor{javadocblue}{rgb}{0.25,0.35,0.75} % javadoc

% Beliebige RGB Farben definieren:
\definecolor{gold}{rgb}{0.83, 0.69, 0.15}
\definecolor{magenta}{rgb}{0.79, 0.08, 0.48}

% Titel in Kopfzeilen
\usepackage{fancyhdr}
\pagestyle{fancy}
\fancyhf{}
\setlength{\headheight}{20pt}

% Seitenumbrueche werden nicht mehr eingerueckt
\setlength{\parindent}{0em}


% % % % % % % % % % % % % % % % % % % % % % % % % % % % % % 
%Variablen
% % % % % % % % % % % % % % % % % % % % % % % % % % % % % % 
\newcommand{\fach}{Objektorientierte Modellierung und Programmierung}
\newcommand{\dokumentenTitel}{Abgabe Uebungsblatt Nr.09}
\newcommand{\Abgabe}{23.06.2020, 12:00 Uhr}
\newcommand{\memberOne}{Marius Birk}
\newcommand{\memberTwo}{Pieter Vogt}


\newcommand{\tutor}{ Florian Brandt }
% % % % % % % % % % % % % % % % % % % % % % % % % % % % % 

% Kopfzeile auf jeder Seite:
\fancyhead[R]{\dokumentenTitel} % Dokument-Titel
\fancyhead[C]{}
\fancyhead[L]{\memberOne, \memberTwo} % Autorennamen
\renewcommand{\headrulewidth}{0.4pt} %obere Trennlinie

% Fußzeile auf jeder Seite:
\fancyfoot[C]{Seite \thepage \ von \ref{TotPages}} %Seitennummer
\renewcommand{\footrulewidth}{0.4pt} %untere Trennlinie

% Nun beginnt das eigentliche Dokument
\begin{document}
	\thispagestyle{plain} % Keine Kopfzeile auf erster Seite, aber Seitenzahl wird angezeigt
	
	\begin{multicols}{2} % Beginnt zweispaltigen Text fuer Header auf erster Seite
		\hspace{-1cm} % Linken Header-Teil 1cm nach links schieben.
		% Tabelle fuer linke Seite vom Header der ersten Seite
		\begin{tabular}{ll} % Mit l werden die Eintraege linksbuendig
			Autoren: & \memberOne \\ % Zwischen jeder Spalte ein & einfuegen
			& \memberTwo \\
% beendet eine Tabellenzeile 
			Tutor: & \tutor \\  
		\end{tabular}
		
		\columnbreak % Nun beginnt die rechte Seite des Headers
		\hspace{-1cm} % Rechten Header-Teil 1cm nach links schieben.
		% Tabelle fuer rechte Seite vom Header der ersten Seite
		\begin{tabular}{ll} % p{1cm} bewirkt, dass die rechte Spalte 6cm breit ist.
			Abgabe: & \Abgabe \\ % Zwischen jeder Spalte ein & einfuege
			Smileys: &  
			%Mit diesem Befehl wird die Zeilenhoehe der folgenden Tabelle um 20% erhoeht.   
			\renewcommand{\arraystretch}{1.2} 
			% Nun kommt eine innere Tabelle in der aeusseren Tabelle, mit der eine Punktetabelle fuer den Tutor erstellt wird:  
			\begin{tabular}{|p{0.8cm}|p{0.8cm}|p{0.8cm}|p{0.8cm}|p{0.8cm}|}
				\hline A1 & A2 & A3 &$\sum\limits^{ }$ \\ \hline
				& & & \\ \hline    
			\end{tabular} \\
		\end{tabular}
		
	\end{multicols} % Beendet zweispaltigen Text
	
	\begin{center}
		\Large{\fach} \\
		\LARGE{\dokumentenTitel} \\
		\small
		$($Alle allgemeinen Definitionen aus der Vorlesung haben in diesem Dokument bestand, es sei den sie erhalten eine explizit andere Definition.$)$
	\end{center}


%~~~~~DOKUMENT ANFANG~~~~~%

%~~~~~Aufgabe 1
\section*{Aufgabe 1}
\begin{lstlisting}
package sample;
import javafx.application.Application;
import javafx.beans.value.ChangeListener;
import javafx.collections.ObservableList;
import javafx.fxml.FXMLLoader;
import javafx.geometry.Insets;
import javafx.scene.Parent;
import javafx.scene.Scene;
import javafx.scene.layout.Background;
import javafx.scene.layout.BackgroundFill;
import javafx.scene.layout.CornerRadii;
import javafx.scene.paint.Color;
import javafx.stage.Stage;
import javafx.scene.shape.Rectangle;
import javafx.scene.control.Slider;
import javafx.scene.layout.VBox;

public class Main extends Application {

    @Override
    public void start(Stage primaryStage) throws Exception{
        Parent root = FXMLLoader.load(getClass().getResource("sample.fxml"));
        //Initialize Components
        Rectangle rect = new Rectangle();
        Slider red = new Slider();
        Slider green = new Slider();
        Slider blue = new Slider();
        VBox vBox = new VBox();
        ObservableList list = vBox.getChildren();

        rect.setHeight(160);
        rect.setWidth(300);

        red.setMin(0);
        red.setMax(255);
        red.setValue(0);
        red.setShowTickLabels(true);
        red.setShowTickMarks(true);
        red.setMajorTickUnit(25);
        red.setMinorTickCount(5);
        red.setBlockIncrement(25);
        red.setBackground(new Background(new BackgroundFill(Color.RED, CornerRadii.EMPTY, Insets.EMPTY)));

        green.setMin(0);
        green.setMax(255);
        green.setValue(0);
        green.setShowTickLabels(true);
        green.setShowTickMarks(true);
        green.setMajorTickUnit(25);
        green.setMinorTickCount(5);
        green.setBlockIncrement(25);
        green.setBackground(new Background(new BackgroundFill(Color.GREEN, CornerRadii.EMPTY, Insets.EMPTY)));

        blue.setMin(0);
        blue.setMax(255);
        blue.setValue(0);
        blue.setShowTickLabels(true);
        blue.setShowTickMarks(true);
        blue.setMajorTickUnit(25);
        blue.setMinorTickCount(5);
        blue.setBlockIncrement(25);
        blue.setBackground(new Background(new BackgroundFill(Color.BLUE, CornerRadii.EMPTY, Insets.EMPTY)));

        ChangeListener<Object> updateListener = (obs, oldValue, newValue) -> {
            int cRed   = (int) red.getValue();
            int cGreen   = (int) green.getValue();
            int cBlue   = (int) blue.getValue();
            rect.setFill(Color.rgb(cRed, cGreen, cBlue));
        };

        red.valueProperty().addListener(updateListener);
        green.valueProperty().addListener(updateListener);
        blue.valueProperty().addListener(updateListener);

        vBox.setSpacing(10);
        vBox.setMargin(rect, new Insets(20,20,20,20));
        vBox.setMargin(red, new Insets(20,20,20,20));
        vBox.setMargin(green, new Insets(20,20,20,20));
        vBox.setMargin(blue, new Insets(20,20,20,20));
        list.addAll(rect, red, green, blue);

        Scene scene = new Scene(vBox);
        primaryStage.setTitle("Color-Mixer");
        primaryStage.setScene(scene);
        primaryStage.show();
    }


    public static void main(String[] args) {
        launch(args);
    }
}

\end{lstlisting}
\section*{Aufgabe 2}
\begin{lstlisting}
	import javafx.application.Application;
import javafx.event.EventHandler;
import javafx.scene.Scene;
import javafx.scene.layout.GridPane;
import javafx.scene.paint.Color;
import javafx.scene.shape.Rectangle;
import javafx.scene.shape.StrokeType;
import javafx.stage.Stage;



public class Lights extends Application {

    public static void main(String[] args) {
        launch(args);
    }
    static int randomNumber(){
        int size = (int)(Math.random()*10);

        while (size < 2){
            size = (int)(Math.random()*10);
        }
        return size;
    }

    @Override
    public void start(Stage primaryStage) {
        int size = randomNumber();
        Rectangle[][] arrRect = new Rectangle[size][size];
        for(int j= 0; j< arrRect.length;j++){
            for(int i = 0; i < arrRect.length; i++)
            {
                Rectangle rect = new Rectangle();
                rect.setFill(Color.WHITE);
                rect.setWidth(100);
                rect.setHeight(100);
                rect.setStrokeType(StrokeType.INSIDE);
                rect.setStroke(Color.BLACK);
                arrRect[i][j] = rect;
            }
        }

        GridPane grid = new GridPane();
        for(int j=0; j<arrRect.length;j++){
            for(int i = 0; i<arrRect.length;i++){
                grid.add(arrRect[j][i], j, i);
            }
        }
        for(int j =0; j<arrRect.length;j++){
            for(int i = 0; i<arrRect.length;i++){
                int finalI = i;
                int finalJ = j;
                arrRect[i][j].setOnMouseClicked(new EventHandler<javafx.scene.input.MouseEvent>() {
                    @Override
                    public void handle(javafx.scene.input.MouseEvent mouseEvent) {
                        if (arrRect[finalI][finalJ].getFill() == Color.YELLOW) {
                            arrRect[finalI][finalJ].setFill(Color.WHITE);

                        } else {
                            arrRect[finalI][finalJ].setFill(Color.YELLOW);
                        }
                        try {
                            if (arrRect[finalI][finalJ + 1].getFill() == Color.YELLOW) {
                                arrRect[finalI][finalJ + 1].setFill(Color.WHITE);
                            } else {
                                arrRect[finalI][finalJ + 1].setFill(Color.YELLOW);
                            }
                            if (arrRect[finalI + 1][finalJ].getFill() == Color.YELLOW) {
                                arrRect[finalI + 1][finalJ].setFill(Color.WHITE);
                            } else {
                                arrRect[finalI + 1][finalJ].setFill(Color.YELLOW);
                            }
                        } catch (ArrayIndexOutOfBoundsException e) {

                        }
                        if (arrRect[finalI][finalJ - 1].getFill() == Color.YELLOW) {
                            arrRect[finalI][finalJ - 1].setFill(Color.WHITE);
                        } else {
                            arrRect[finalI][finalJ - 1].setFill(Color.YELLOW);
                        }
                        if (finalI > 1) {
                            if (arrRect[finalI - 1][finalJ].getFill() == Color.YELLOW) {
                                arrRect[finalI - 1][finalJ].setFill(Color.WHITE);
                            } else {
                                arrRect[finalI - 1][finalJ].setFill(Color.YELLOW);
                            }
                        }
                    }
                });
            }
        }


        Scene scene = new Scene(grid, 200, 100);

        primaryStage.setHeight((double) size*arrRect[0][0].getHeight()+60);
        primaryStage.setWidth((double) size*arrRect[0][0].getWidth()+40);
        primaryStage.setTitle("Lights");
        primaryStage.setScene(scene);
        primaryStage.show();
    }
}

\end{lstlisting}
\section*{Aufgabe 3}
\subsection*{Captain}
\begin{lstlisting}
	public abstract class Captain {
	
		protected Ship ship;
		
		public Captain(Ship ship) {
			super();
			this.ship = ship;
		}
	
		/**
		 * Gibt ein Kommando an das Schiff.
		 * Dieses Kommando wird erst auf der Konsole ausgegeben
		 * und anschliessend wird die entsprechende Methode des
		 * Schiffs aufgerufen.
		 */
		public abstract void commandShip();
	
	}
	
\end{lstlisting}
\\ 
\subsection*{Klasse Observable}
\begin{lstlisting}
	import java.lang.reflect.Array;
	import java.util.ArrayList;
	
	public abstract class Observable implements Observer{
		private ArrayList<Observer> observers;
		public Observable() {
			observers = new ArrayList<>();
		}
	
		public void addObserver(Observer obs){
			observers.add(obs);
		}
		public void removeObserver(Observer obs){
			observers.remove(obs);
		}
	
		public void setChanged(ShipEvent what){
	
		}
		public void clearChanged(){
	
		}
		public boolean isChanged(){
			return true;
		}
		public void notifyObservers(ShipEvent what){
			for(Observer o : observers){
				o.update(this, what);
			}
		}
		@Override
		public void update(Observable who, ShipEvent what) {
			who.setChanged(what);
		}
	}

\end{lstlisting}
\subsection*{Klasse Ship}
\begin{lstlisting}
	public class Ship extends Observable{
    private ShipEvent what;
    public void setShipEvent(ShipEvent what){
        this.what = what;
        notifyObservers(what);
    }
}
\end{lstlisting}
\subsection*{Drunken Pirate}
\begin{lstlisting}
	import java.beans.PropertyChangeListener;
import java.util.Properties;
import java.util.Random;

public class DrunkenPirate extends Captain{
    private int lastPick=0;
    private boolean cannon = false;
    private boolean sails = false;


    public DrunkenPirate(Ship ship) {
        super(ship);
    }

    @Override
    public void commandShip() {
        int pick = new Random().nextInt(ShipEvent.values().length);
        while(pick==lastPick){
            pick = new Random().nextInt(ShipEvent.values().length);
        }
        if(ShipEvent.values()[pick].equals(ShipEvent.SET_SAILS)){
            if(sails == false){
                System.out.println(ShipEvent.values()[pick]);
                sails= true;
            }
        }else if(ShipEvent.values()[pick].equals(ShipEvent.STRIKE_SAILS)){
            if(sails==true){
                System.out.println(ShipEvent.values()[pick]);
                sails = false;
            }
        }else if(ShipEvent.values()[pick].equals(ShipEvent.LOAD_CANNONS)){
            if(cannon==false){
                System.out.println(ShipEvent.values()[pick]);
                cannon=true;
            }
        }else if(ShipEvent.values()[pick].equals(ShipEvent.FIRE_CANNONS)){
            if(cannon==true){
                System.out.println(ShipEvent.values()[pick]);
                cannon=false;
            }
        }else if(ShipEvent.values()[pick].equals(ShipEvent.NO_EVENT)||ShipEvent.values()[pick].equals(ShipEvent.TURN_LEFT)||ShipEvent.values()[pick].equals(ShipEvent.TURN_RIGHT)){
            System.out.println(ShipEvent.values()[pick]);
        }
        lastPick = pick;
    }

}
\end{lstlisting}
\subsection*{ShipLog}
\begin{lstlisting}
	public class ShipLog implements Observer{
    @Override
    public void update(Observable who, ShipEvent what) {
    }
}
\end{lstlisting}
\end{document}

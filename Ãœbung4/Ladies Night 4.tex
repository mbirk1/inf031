\documentclass[12pt,a4paper,oneside,ngerman]{article} 
\usepackage[left=3cm,right=3cm,top=2.5cm]{geometry} % Groesse der Seitenraender definieren
\usepackage[utf8]{inputenc} % utf8 encoding
\usepackage{hyperref}
\usepackage{ngerman}[babel]
\usepackage{graphicx}
\usepackage{tikz} % Automaten, Graphen, ... zeichnen
\usepackage{tikz-qtree} % Paket fuer Tikz Graphen-Baeume
\usetikzlibrary{arrows,shapes,automata} % Bestimmte tikz-Befehle benutzen
\usepackage{amsmath,amssymb} % Mathe-Formeln und -Ausdruecke
\usepackage{listings} % Code-Ausschnitte einbinden
\usepackage{xcolor} % Eigene Farben definieren
\usepackage{colortbl} % Farben verwenden in Tabellen
\usepackage{wrapfig} % Bilder von Text umfliessen lassen
\usepackage{multicol} % Mehrspaltigen Text schreiben
\usepackage{stmaryrd} % Fuer Symbole wie zu Beispiel Widerspruchspfeil
\usepackage{caption}
\usepackage{totpages}

\usepackage{circuitikz}
\usepackage{amsmath}
\usepackage{graphicx}
\usepackage{subfigure}
\usepackage{tikz}
\usepackage{float}

\lstset{language=Java,
basicstyle=\small \ttfamily,
keywordstyle=\color{javapurple}\bfseries,
stringstyle=\color{javared},
commentstyle=\color{javagreen},
morecomment=[s][\color{javadocblue}]{/**}{*/},
numbers=left,
numberstyle=\tiny\color{black},
stepnumber=1,
numbersep=10pt,
tabsize=1,
showspaces=false,
showstringspaces=false,
breaklines=true}
\lstset{literate=%
  {Ö}{{\"O}}1
  {Ä}{{\"A}}1
  {Ü}{{\"U}}1
  {ß}{{\ss}}1
  {ü}{{\"u}}1
  {ä}{{\"a}}1
  {ö}{{\"o}}1
}

% Beliebige RGB Farben definieren:
\definecolor{gold}{rgb}{0.83, 0.69, 0.15}
\definecolor{magenta}{rgb}{0.79, 0.08, 0.48}
\definecolor{javared}{rgb}{0.6,0,0} % for strings
\definecolor{javagreen}{rgb}{0.25,0.5,0.35} % comments
\definecolor{javapurple}{rgb}{0.5,0,0.35} % keywords
\definecolor{javadocblue}{rgb}{0.25,0.35,0.75} % javadoc

% Beliebige RGB Farben definieren:
\definecolor{gold}{rgb}{0.83, 0.69, 0.15}
\definecolor{magenta}{rgb}{0.79, 0.08, 0.48}

% Titel in Kopfzeilen
\usepackage{fancyhdr}
\pagestyle{fancy}
\fancyhf{}
\setlength{\headheight}{20pt}

% Seitenumbrueche werden nicht mehr eingerueckt
\setlength{\parindent}{0em}


% % % % % % % % % % % % % % % % % % % % % % % % % % % % % % 
%Variablen
% % % % % % % % % % % % % % % % % % % % % % % % % % % % % % 
\newcommand{\fach}{Objektorientierte Modellierung und Programmierung}
\newcommand{\dokumentenTitel}{Abgabe Uebungsblatt Nr.04}
\newcommand{\Abgabe}{19.05.2020, 12:00 Uhr}
\newcommand{\memberOne}{Marius Birk}
\newcommand{\memberTwo}{Pieter Vogt}


\newcommand{\tutor}{ Florian Brandt }
% % % % % % % % % % % % % % % % % % % % % % % % % % % % % 

% Kopfzeile auf jeder Seite:
\fancyhead[R]{\dokumentenTitel} % Dokument-Titel
\fancyhead[C]{}
\fancyhead[L]{\memberOne, \memberTwo} % Autorennamen
\renewcommand{\headrulewidth}{0.4pt} %obere Trennlinie

% Fußzeile auf jeder Seite:
\fancyfoot[C]{Seite \thepage \ von \ref{TotPages}} %Seitennummer
\renewcommand{\footrulewidth}{0.4pt} %untere Trennlinie

% Nun beginnt das eigentliche Dokument
\begin{document}
	\thispagestyle{plain} % Keine Kopfzeile auf erster Seite, aber Seitenzahl wird angezeigt
	
	\begin{multicols}{2} % Beginnt zweispaltigen Text fuer Header auf erster Seite
		\hspace{-1cm} % Linken Header-Teil 1cm nach links schieben.
		% Tabelle fuer linke Seite vom Header der ersten Seite
		\begin{tabular}{ll} % Mit l werden die Eintraege linksbuendig
			Autoren: & \memberOne \\ % Zwischen jeder Spalte ein & einfuegen
			& \memberTwo \\
% beendet eine Tabellenzeile 
			Tutor: & \tutor \\  
		\end{tabular}
		
		\columnbreak % Nun beginnt die rechte Seite des Headers
		\hspace{-1cm} % Rechten Header-Teil 1cm nach links schieben.
		% Tabelle fuer rechte Seite vom Header der ersten Seite
		\begin{tabular}{ll} % p{1cm} bewirkt, dass die rechte Spalte 6cm breit ist.
			Abgabe: & \Abgabe \\ % Zwischen jeder Spalte ein & einfuege
			Smileys: &  
			%Mit diesem Befehl wird die Zeilenhoehe der folgenden Tabelle um 20% erhoeht.   
			\renewcommand{\arraystretch}{1.2} 
			% Nun kommt eine innere Tabelle in der aeusseren Tabelle, mit der eine Punktetabelle fuer den Tutor erstellt wird:  
			\begin{tabular}{|p{0.8cm}|p{0.8cm}|p{0.8cm}|p{0.8cm}|p{0.8cm}|}
				\hline A1 & A2 & A3 & A4 &$\sum\limits^{ }$ \\ \hline
				& & & & \\ \hline    
			\end{tabular} \\
		\end{tabular}
		
	\end{multicols} % Beendet zweispaltigen Text
	
	\begin{center}
		\Large{\fach} \\
		\LARGE{\dokumentenTitel} \\
		\small
		$($Alle allgemeinen Definitionen aus der Vorlesung haben in diesem Dokument bestand, es sei den sie erhalten eine explizit andere Definition.$)$
	\end{center}
\section*{Aufgabe 1}
   \begin{lstlisting}
      import java.lang.Math;
//Dieser Code läuft für das explizite Beispiel aus der Aufgabe.
public class Functions {
    public static void main(String[]  args){
        double i = 2;
        Function chain = new SineFunction(new SquareFunction());
        chain.calculate(i);
    }

}
interface Function{
    void calculate(double i);
    double  result=0;
}
class SineFunction implements Function{
    private String function =" ";
    public SineFunction(SquareFunction squareFunction) {
        if(squareFunction.getClass()== SquareFunction.class){
            function="square";
        }
        else{
            function="sine";
        }
        //Diese If Abfragen können bei beliebiger Reihenfolge der Funktionen entsprechend erweitert werden.
        //Weitere Anpassungen am Code sind dann aber noch nötig.
    }
    @Override public void calculate(double i ){
        if(function.equals("sqaure")){
            i=i*i;
        }
        i=Math.sin(i);
        System.out.println(i);
    }
}
class SquareFunction implements Function{
    @Override
    public void calculate(double i) {

    }
}
\end{lstlisting}
\section{Aufgabe 2}
   \subsection*{Teilaufgabe a}
      \begin{lstlisting}
   interface Sequence{
      public int current=0;
  }  
      \end{lstlisting}
   \subsection*{Teilaufgabe b}
      \begin{lstlisting}
         public class Naturals implements Sequence {
            int current = 0;
         
            @Override
            public int getNext() {
               current++;
               return current;
            }
         }
      \end{lstlisting}
   \subsection*{Teilaufgabe c}
      \begin{lstlisting}
         abstract class Filter implements Sequence1{
            public Filter(Sequence1 sequence) {
                Object s = sequence;
            }
        }        
      \end{lstlisting}
   \subsection*{Teilaufgabe d}
      \begin{lstlisting}
         class ZapMultiples extends Filter{
            private int cur=0;
            private int basic;
            public ZapMultiples(int base$,$ Sequence1 sequence) {
                super(sequence);
                basic = base;
            }
        
            @Override public int current(){
                return cur;
            }
            @Override public int getNext(){
                cur++;
                if(cur%basic ==0){
                    cur=cur+1;
                    return cur;
                }else{
                    return cur;
                }
            }
        }     
      \end{lstlisting}
   \subsection*{Teilaufgabe e}
      \begin{lstlisting}
         import java.util.ArrayList;

public class Primes implements Sequence {

   Sequence sequence = new Naturals();
   int next = 1;
   ArrayList<Integer> primes = new ArrayList<>();

   private void incNext() {
      next++;
   }

   @Override
   public int getNext() {
      if (checkIfPrime()) {
         primes.add(next);
         incNext();
         return sequence.getNext();
      } else {
         incNext();
         sequence.getNext();
         return getNext();
      }
      /*if (checkIfPrime(next) && next != 1) {
         incNext();
         return sequence.getNext();
      } else {
         incNext();
         sequence.getNext();
         return getNext();
      }*/
   }

   private boolean checkIfPrime() {
      if (next == 1) {
         return false;
      }
      if (isDividableByPrim(next)) {
         return false;
      } else return true;
      /*
      for (int i = 2; i < number; i++) {
         if (next % i == 0) {
            return false;
         }
      }
      return true;
      */
   }

   private boolean isDividableByPrim(int next) {
      for (Integer i : primes) {
         if (next % i == 0) {
            return true;
         }
      }
      return false;
   }

   public Primes() {
   }
}
\end{lstlisting}
\section{Aufgabe 3}
\begin {lstlisting}
import jdk.jshell.execution.Util;

import javax.print.DocFlavor;

public class Compare {
    public static void main(String[] args){
        Object[] objects = new Object[3];
        Comparable one = new ComparableInteger(1);
        Comparable four = new ComparableInteger(4);
        Comparable seven = new ComparableInteger(7);

        objects[0] = one;
        objects[1] = four;
        objects[2] = seven;

        Utils util = new Utils();
        Comparable getMinimum;
    }
}
interface Comparable{
    public int compareTo(Comparable obj);
    public int getValue();
    public void setValue(int i);
}

class Utils{
    public static Comparable getMinimum(Comparable[] elements){
        ComparableInteger min= new ComparableInteger(0);
        for(int i =0; i<elements.length;i++){

            if (min.getValue() > elements[i].getValue()) {
                min.setValue(i);
            }
        }
        return elements[min.getValue()];
    }

}
class Integer{
    protected int value;
    public Integer(int value){
        this.value=value;
    }
    public int getValue(){
        return value;
    }
}
class ComparableInteger implements Comparable{
    protected int value;
    public ComparableInteger(int value){
        this.value=value;
    }
    public int getValue(){
        return value;
    }

    @Override public void setValue(int value) {
        this.value = value;
    }

    @Override
    public int compareTo(Comparable obj) {
        return 0;
    }
}
\end{lstlisting}

\section{Aufgabe 4}
\begin{lstlisting}
   import java.util.ArrayList;

public class Rooms {
    public static void main(String[] args){
        Furniture chair = new Chair();
        Desk desk = new Desk();
        Chair chair1 = new Chair();

        Office office = new Office(desk, chair1, chair);
    }
}
interface Furniture{

}
class Table implements Furniture{
    private static int legs = 4;
    public int getLegs(){
        return legs;
    }
}
class Desk extends Table{

}
class Chair implements Furniture{
    private static int legs = 4;
    public int getLegs(){
        return legs;
    }
}
class Office{
    private int i=0;
    ArrayList<Chair> chairs = new ArrayList<Chair>();
    ArrayList<Desk> desks = new ArrayList<Desk>();
    ArrayList<Furniture> furniture = new ArrayList<Furniture>();

    public Office( Desk desk, Chair chair, Furniture fur) {
        desks.add(desk);
        chairs.add(chair);
        furniture.add(fur);
    }
}

class Room extends Office{
    public Room(Desk desk, Chair chair, Furniture fur) {
        super(desk, chair, fur);
    }
}
\end{lstlisting}
\end{document}

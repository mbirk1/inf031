\documentclass[12pt,a4paper,oneside,ngerman]{article} 
\usepackage[left=3cm,right=3cm,top=2.5cm]{geometry} % Groesse der Seitenraender definieren
\usepackage[utf8]{inputenc} % utf8 encoding
\usepackage{hyperref}
\usepackage{ngerman}[babel]
\usepackage{graphicx}
\usepackage{tikz} % Automaten, Graphen, ... zeichnen
\usepackage{tikz-qtree} % Paket fuer Tikz Graphen-Baeume
\usetikzlibrary{arrows,shapes,automata} % Bestimmte tikz-Befehle benutzen
\usepackage{amsmath,amssymb} % Mathe-Formeln und -Ausdruecke
\usepackage{listings} % Code-Ausschnitte einbinden
\usepackage{xcolor} % Eigene Farben definieren
\usepackage{colortbl} % Farben verwenden in Tabellen
\usepackage{wrapfig} % Bilder von Text umfliessen lassen
\usepackage{multicol} % Mehrspaltigen Text schreiben
\usepackage{stmaryrd} % Fuer Symbole wie zu Beispiel Widerspruchspfeil
\usepackage{caption}
\usepackage{totpages}

\usepackage{circuitikz}
\usepackage{amsmath}
\usepackage{graphicx}
\usepackage{subfigure}
\usepackage{tikz}
\usepackage{float}

\lstset{language=Java,
basicstyle=\small \ttfamily,
keywordstyle=\color{javapurple}\bfseries,
stringstyle=\color{javared},
commentstyle=\color{javagreen},
morecomment=[s][\color{javadocblue}]{/**}{*/},
numbers=left,
numberstyle=\tiny\color{black},
stepnumber=1,
numbersep=10pt,
tabsize=1,
showspaces=false,
showstringspaces=false,
breaklines=true}
\lstset{literate=%
  {Ö}{{\"O}}1
  {Ä}{{\"A}}1
  {Ü}{{\"U}}1
  {ß}{{\ss}}1
  {ü}{{\"u}}1
  {ä}{{\"a}}1
  {ö}{{\"o}}1
}

% Beliebige RGB Farben definieren:
\definecolor{gold}{rgb}{0.83, 0.69, 0.15}
\definecolor{magenta}{rgb}{0.79, 0.08, 0.48}
\definecolor{javared}{rgb}{0.6,0,0} % for strings
\definecolor{javagreen}{rgb}{0.25,0.5,0.35} % comments
\definecolor{javapurple}{rgb}{0.5,0,0.35} % keywords
\definecolor{javadocblue}{rgb}{0.25,0.35,0.75} % javadoc

% Beliebige RGB Farben definieren:
\definecolor{gold}{rgb}{0.83, 0.69, 0.15}
\definecolor{magenta}{rgb}{0.79, 0.08, 0.48}

% Titel in Kopfzeilen
\usepackage{fancyhdr}
\pagestyle{fancy}
\fancyhf{}
\setlength{\headheight}{20pt}

% Seitenumbrueche werden nicht mehr eingerueckt
\setlength{\parindent}{0em}


% % % % % % % % % % % % % % % % % % % % % % % % % % % % % % 
%Variablen
% % % % % % % % % % % % % % % % % % % % % % % % % % % % % % 
\newcommand{\fach}{Objektorientierte Modellierung und Programmierung}
\newcommand{\dokumentenTitel}{Abgabe Uebungsblatt Nr.10}
\newcommand{\Abgabe}{07.07.2020, 12:00 Uhr}
\newcommand{\memberOne}{Marius Birk}
\newcommand{\memberTwo}{Pieter Vogt}


\newcommand{\tutor}{ Florian Brandt }
% % % % % % % % % % % % % % % % % % % % % % % % % % % % % 

% Kopfzeile auf jeder Seite:
\fancyhead[R]{\dokumentenTitel} % Dokument-Titel
\fancyhead[C]{}
\fancyhead[L]{\memberOne, \memberTwo} % Autorennamen
\renewcommand{\headrulewidth}{0.4pt} %obere Trennlinie

% Fußzeile auf jeder Seite:
\fancyfoot[C]{Seite \thepage \ von \ref{TotPages}} %Seitennummer
\renewcommand{\footrulewidth}{0.4pt} %untere Trennlinie

% Nun beginnt das eigentliche Dokument
\begin{document}
	\thispagestyle{plain} % Keine Kopfzeile auf erster Seite, aber Seitenzahl wird angezeigt
	
	\begin{multicols}{2} % Beginnt zweispaltigen Text fuer Header auf erster Seite
		\hspace{-1cm} % Linken Header-Teil 1cm nach links schieben.
		% Tabelle fuer linke Seite vom Header der ersten Seite
		\begin{tabular}{ll} % Mit l werden die Eintraege linksbuendig
			Autoren: & \memberOne \\ % Zwischen jeder Spalte ein & einfuegen
			& \memberTwo \\
% beendet eine Tabellenzeile 
			Tutor: & \tutor \\  
		\end{tabular}
		
		\columnbreak % Nun beginnt die rechte Seite des Headers
		\hspace{-1cm} % Rechten Header-Teil 1cm nach links schieben.
		% Tabelle fuer rechte Seite vom Header der ersten Seite
		\begin{tabular}{ll} % p{1cm} bewirkt, dass die rechte Spalte 6cm breit ist.
			Abgabe: & \Abgabe \\ % Zwischen jeder Spalte ein & einfuege
			Smileys: &  
			%Mit diesem Befehl wird die Zeilenhoehe der folgenden Tabelle um 20% erhoeht.   
			\renewcommand{\arraystretch}{1.2} 
			% Nun kommt eine innere Tabelle in der aeusseren Tabelle, mit der eine Punktetabelle fuer den Tutor erstellt wird:  
			\begin{tabular}{|p{0.8cm}|p{0.8cm}|p{0.8cm}|p{0.8cm}|p{0.8cm}|}
				\hline A1 & A2 & A3 &$\sum\limits^{ }$ \\ \hline
				& & & \\ \hline    
			\end{tabular} \\
		\end{tabular}
		
	\end{multicols} % Beendet zweispaltigen Text
	
	\begin{center}
		\Large{\fach} \\
		\LARGE{\dokumentenTitel} \\
		\small
		$($Alle allgemeinen Definitionen aus der Vorlesung haben in diesem Dokument bestand, es sei den sie erhalten eine explizit andere Definition.$)$
	\end{center}


%~~~~~DOKUMENT ANFANG~~~~~%

%~~~~~Aufgabe 1
\section*{Aufgabe 1}
\begin{lstlisting}
class Output extends Thread {
	public void run() {
		synchronized (InOut.getLock()) {
			if (!InOut.isEntered()) {
				try {
					InOut.getLock().wait();
				} catch (InterruptedException e) {
				}
			}
			System.out.println(InOut.getValue() * InOut.getValue());
		}
	}
}

class Input extends Thread {

	public void run() {
		synchronized (InOut.getLock()) {
			InOut.setValue(IO.readInt("Value: "));
			InOut.setEntered(true);
			InOut.getLock().notify();
		}
	}
}

public class InOut {

	private static Object lock = new Object();
	private static boolean entered = false;
	private static int value = 0;
	
	public static Object getLock() {
		return lock;
	}
	
	public static boolean isEntered() {
		return entered;
	}
	
	public static void setEntered(boolean entered) {
		InOut.entered = entered;
	}
	
	public static int getValue() {
		return value;
	}
	
	public static void setValue(int value) {
		InOut.value = value;
	}

	public static void main(String[] args) {
		new Output().start();
		new Input().start();
	}

}

import java.util.concurrent.Semaphore;
class OutputThreaded extends Thread {
    static Semaphore semaphore = new Semaphore(1);
    public void run() {
        try {
            semaphore.acquire();
            System.out.println(InOut.getValue() * InOut.getValue());
        } catch (InterruptedException e) {
            e.printStackTrace();
        }finally {
            semaphore.release();
        }
    }
}

class InputThreaded extends Thread {
    static Semaphore semaphore = new Semaphore(1);
    public void run() {
        try {
            semaphore.acquire();
            InOut.setValue(IO.readInt("Value: "));
            InOut.setEntered(true);
        } catch (InterruptedException e) {
            e.printStackTrace();
        }finally {
            semaphore.release();
        }
    }
}public class InOutThreaded {
    private static Object lock = new Object();
    private static boolean entered = false;
    private static int value = 0;
    public static boolean isEntered() {
            return entered;
        }
    public static void setEntered(boolean entered) {
            InOutThreaded.entered = entered;
        }
    public static int getValue() {
            return value;
        }
    public static void setValue(int value) {
            InOutThreaded.value = value;
        }
    public static void main(String[] args) {

        new Output().start();
        new Input().start();
    }
}
\end{lstlisting}
\section*{Aufgabe 2}
\begin{lstlisting}
	import java.util.concurrent.BrokenBarrierException;
	import java.util.concurrent.CyclicBarrier;
	
	public class Barriers {
	
		private final static int NUMBER = 3;
		public static CyclicBarrier barrier = new CyclicBarrier(NUMBER);
	
		public static void main(String[] args) {
			NumberRunner[] runner = new NumberRunner[NUMBER];
			for (int i = 0; i < NUMBER; i++) {
				runner[i] = new NumberRunner(i);
			}
			for (int i = 0; i < NUMBER; i++) {
				runner[i].start();
			}
		}
	
	}
	class NumberRunner extends Thread {
	
		private int number;
	
		public NumberRunner(int n) {
			number = n;
		}
		@Override
		public void run() {
			for (int i = 0; i < 100; i++) {
				for(int j=0; j<10;j++){
					System.out.println("Thread " + number + ": " + i+j);
				}
				if(i%10==0){
					try {
						Barriers.barrier.await();
					} catch (InterruptedException e) {
						e.printStackTrace();
					} catch (BrokenBarrierException e) {
						e.printStackTrace();
					}
				}
	
				try {
					if(Barriers.barrier.await()==3) {
						Barriers.barrier.reset();
					}
				} catch (InterruptedException e) {
					e.printStackTrace();
				} catch (BrokenBarrierException e) {
					e.printStackTrace();
				}
			}
			//in der aufgabe steht jeweils 10 ausgaben. Dürfen es auch weniger als 10 sein solange es nicht mehr als 10 sind?
		}
	}		
\end{lstlisting}

\section*{Aufgabe 3}
\begin{lstlisting}
    import java.util.Collection;
import java.util.Collections;
import java.util.Comparator;

public class KnapsackRecursive extends Knapsack {

	public KnapsackRecursive(int capacity, Collection<Item> candidates) {
		super(capacity, candidates);
	}

	@Override
	public Selection pack() {
		return new Selection();
	}

}

import java.util.Collection;
import java.util.HashMap;
import java.util.Map;

public class KnapsackDynamic extends Knapsack {
	
	public KnapsackDynamic(int capacity, Collection<Item> candidates) {
		super(capacity, candidates);
	}

	@Override
	public Selection pack() {
		//TODO: implement this
		return new Selection();
	}

}

import java.util.ArrayList;
import java.util.Collection;
import java.util.Collections;
import java.util.Comparator;

public class KnapsackGreedy extends Knapsack {

	public KnapsackGreedy(int capacity, Collection<Item> candidates) {
		super(capacity, candidates);
	}

	@Override
	public Selection pack() {
		Selection select = new Selection();
		Selection test = new Selection();
		Collections.sort(candidates, new Comparator<Item>() {
			@Override
			public int compare(Item o1, Item o2) {
				return o1.getValue()-o2.getValue();
			}
		});
		Collections.reverse(candidates);
		while(capacity!=0){
			if(capacity>=30){
				select = new Selection(test, candidates.get(1));
				capacity=capacity-30;
			}else if(capacity<30 && capacity>= 5){
				select = new Selection(select, candidates.get(2));
				capacity=capacity-5;
			}else if(capacity<5 && capacity>=1){
				select = new Selection(select, candidates.get(0));
				capacity=capacity-1;
			}
		}
		return new Selection(select);
	}
}

\end{lstlisting}
\end{document}

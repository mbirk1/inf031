\documentclass[12pt,a4paper,oneside,ngerman]{article} 
\usepackage[left=3cm,right=3cm,top=2.5cm]{geometry} % Groesse der Seitenraender definieren
\usepackage[utf8]{inputenc} % utf8 encoding
\usepackage{hyperref}
\usepackage{ngerman}[babel]
\usepackage{graphicx}
\usepackage{tikz} % Automaten, Graphen, ... zeichnen
\usepackage{tikz-qtree} % Paket fuer Tikz Graphen-Baeume
\usetikzlibrary{arrows,shapes,automata} % Bestimmte tikz-Befehle benutzen
\usepackage{amsmath,amssymb} % Mathe-Formeln und -Ausdruecke
\usepackage{listings} % Code-Ausschnitte einbinden
\usepackage{xcolor} % Eigene Farben definieren
\usepackage{colortbl} % Farben verwenden in Tabellen
\usepackage{wrapfig} % Bilder von Text umfliessen lassen
\usepackage{multicol} % Mehrspaltigen Text schreiben
\usepackage{stmaryrd} % Fuer Symbole wie zu Beispiel Widerspruchspfeil
\usepackage{caption}
\usepackage{totpages}

\usepackage{circuitikz}
\usepackage{amsmath}
\usepackage{graphicx}
\usepackage{subfigure}
\usepackage{tikz}
\usepackage{float}

\lstset{language=Java,
basicstyle=\small \ttfamily,
keywordstyle=\color{javapurple}\bfseries,
stringstyle=\color{javared},
commentstyle=\color{javagreen},
morecomment=[s][\color{javadocblue}]{/**}{*/},
numbers=left,
numberstyle=\tiny\color{black},
stepnumber=1,
numbersep=10pt,
tabsize=1,
showspaces=false,
showstringspaces=false,
breaklines=true}
\lstset{literate=%
  {Ö}{{\"O}}1
  {Ä}{{\"A}}1
  {Ü}{{\"U}}1
  {ß}{{\ss}}1
  {ü}{{\"u}}1
  {ä}{{\"a}}1
  {ö}{{\"o}}1
}

% Beliebige RGB Farben definieren:
\definecolor{gold}{rgb}{0.83, 0.69, 0.15}
\definecolor{magenta}{rgb}{0.79, 0.08, 0.48}
\definecolor{javared}{rgb}{0.6,0,0} % for strings
\definecolor{javagreen}{rgb}{0.25,0.5,0.35} % comments
\definecolor{javapurple}{rgb}{0.5,0,0.35} % keywords
\definecolor{javadocblue}{rgb}{0.25,0.35,0.75} % javadoc

% Beliebige RGB Farben definieren:
\definecolor{gold}{rgb}{0.83, 0.69, 0.15}
\definecolor{magenta}{rgb}{0.79, 0.08, 0.48}

% Titel in Kopfzeilen
\usepackage{fancyhdr}
\pagestyle{fancy}
\fancyhf{}
\setlength{\headheight}{20pt}

% Seitenumbrueche werden nicht mehr eingerueckt
\setlength{\parindent}{0em}


% % % % % % % % % % % % % % % % % % % % % % % % % % % % % % 
%Variablen
% % % % % % % % % % % % % % % % % % % % % % % % % % % % % % 
\newcommand{\fach}{Objektorientierte Modellierung und Programmierung}
\newcommand{\dokumentenTitel}{Abgabe Uebungsblatt Nr.05}
\newcommand{\Abgabe}{26.05.2020, 12:00 Uhr}
\newcommand{\memberOne}{Marius Birk}
\newcommand{\memberTwo}{Pieter Vogt}


\newcommand{\tutor}{ Florian Brandt }
% % % % % % % % % % % % % % % % % % % % % % % % % % % % % 

% Kopfzeile auf jeder Seite:
\fancyhead[R]{\dokumentenTitel} % Dokument-Titel
\fancyhead[C]{}
\fancyhead[L]{\memberOne, \memberTwo} % Autorennamen
\renewcommand{\headrulewidth}{0.4pt} %obere Trennlinie

% Fußzeile auf jeder Seite:
\fancyfoot[C]{Seite \thepage \ von \ref{TotPages}} %Seitennummer
\renewcommand{\footrulewidth}{0.4pt} %untere Trennlinie

% Nun beginnt das eigentliche Dokument
\begin{document}
	\thispagestyle{plain} % Keine Kopfzeile auf erster Seite, aber Seitenzahl wird angezeigt
	
	\begin{multicols}{2} % Beginnt zweispaltigen Text fuer Header auf erster Seite
		\hspace{-1cm} % Linken Header-Teil 1cm nach links schieben.
		% Tabelle fuer linke Seite vom Header der ersten Seite
		\begin{tabular}{ll} % Mit l werden die Eintraege linksbuendig
			Autoren: & \memberOne \\ % Zwischen jeder Spalte ein & einfuegen
			& \memberTwo \\
% beendet eine Tabellenzeile 
			Tutor: & \tutor \\  
		\end{tabular}
		
		\columnbreak % Nun beginnt die rechte Seite des Headers
		\hspace{-1cm} % Rechten Header-Teil 1cm nach links schieben.
		% Tabelle fuer rechte Seite vom Header der ersten Seite
		\begin{tabular}{ll} % p{1cm} bewirkt, dass die rechte Spalte 6cm breit ist.
			Abgabe: & \Abgabe \\ % Zwischen jeder Spalte ein & einfuege
			Smileys: &  
			%Mit diesem Befehl wird die Zeilenhoehe der folgenden Tabelle um 20% erhoeht.   
			\renewcommand{\arraystretch}{1.2} 
			% Nun kommt eine innere Tabelle in der aeusseren Tabelle, mit der eine Punktetabelle fuer den Tutor erstellt wird:  
			\begin{tabular}{|p{0.8cm}|p{0.8cm}|p{0.8cm}|p{0.8cm}|p{0.8cm}|}
				\hline A1 & A2 & A3 &$\sum\limits^{ }$ \\ \hline
				& & & \\ \hline    
			\end{tabular} \\
		\end{tabular}
		
	\end{multicols} % Beendet zweispaltigen Text
	
	\begin{center}
		\Large{\fach} \\
		\LARGE{\dokumentenTitel} \\
		\small
		$($Alle allgemeinen Definitionen aus der Vorlesung haben in diesem Dokument bestand, es sei den sie erhalten eine explizit andere Definition.$)$
	\end{center}


%~~~~~DOKUMENT ANFANG~~~~~%

%~~~~~Aufgabe 1
\section*{Aufgabe 1}
\begin{lstlisting}
%%hier dein code
\end{lstlisting}

%~~~~~Aufgabe 2
\section*{Aufgabe 2}
\subsection*{Bill.java}
\begin{lstlisting}
import java.util.ArrayList;

public class Bill {

   //fields

   String name;
   double billPrice = 0;
   ArrayList<BillItem> items = new ArrayList<>();

   //methods

   public void add(CarPart part) {
      items.add(new BillItem(part));
   }

   //getter - setter

   public double getTotalPrice() {
      return billPrice;
   }

   public String toString() {
      StringBuffer tempString = new StringBuffer("Receipt for Bill: ");
      double receipTotal = 0;
      tempString.append(this.name);
      tempString.append("\n");
      for (int i = 0; i < items.size(); i++) {
         tempString.append(items.get(i).item.getName());//add ItemName
         tempString.append("\t");
         tempString.append(items.get(i).item.getPrice());//add ItemPrice
         tempString.append("\n");
         receipTotal = receipTotal + items.get(i).item.getPrice();
      }
      tempString.append("\n");
      Math.nextUp(receipTotal);//doesn't work for some reason.
      tempString.append("In Total this receipt is: " + receipTotal);
      String output = tempString.toString();
      return output;
   }

   //constructors

   public Bill(String name) {
      this.name = name;
   }

   //nested classes

   private class BillItem {

      //fields

      CarPart item;

      //methods

      //getter - setter

      public CarPart getItem() {
         return item;
      }

      public void setItem(CarPart item) {
         this.item = item;
      }

      public BillItem(CarPart item) {
         this.item = item;
      }
   }

}

\end{lstlisting}

\subsection*{Car.java}
\begin{lstlisting}
import java.util.ArrayList;

public class Car {
   ArrayList<CarPart> parts = new ArrayList<>();
}

\end{lstlisting}

\subsection*{CarComponent.java}
\begin{lstlisting}
public interface CarComponent {
   public String getName();
}

\end{lstlisting}

\subsection*{CarPart.java}
\begin{lstlisting}
public class CarPart implements CarComponent {
   String name;
   double price;

   @Override
   public String getName() {
      return null;
   }

   public double getPrice() {
      return price;
   }

   public static class Seat extends CarPart {
      String name = new String("Seat");
      double price = 2000.0;

      @Override
      public String getName() {
         return name;
      }

      public double getPrice() {
         return price;
      }
   }

   public static class Wheel extends CarPart {
      String name = new String("Wheel");

      double price = 1000.0;

      @Override
      public String getName() {
         return name;
      }

      public double getPrice() {
         return price;
      }
   }

   public static class Motor extends CarPart {
      String name = new String("Motor");

      double price = 100000;

      @Override
      public String getName() {
         return name;
      }

      public double getPrice() {
         return price;
      }
   }
}

\end{lstlisting}

\subsection*{Main.java}
\begin{lstlisting}
public class Main {
   public static void main(String[] args) {
      Bill bill = new Bill("Rolls Royce");
      bill.add(new CarPart.Motor());
      bill.add(new CarPart.Seat());
      bill.add(new CarPart.Wheel());
      bill.add(new CarPart.Wheel());
      bill.add(new CarPart.Wheel());
      bill.add(new CarPart.Wheel());
      System.out.println(bill.toString());
   }
}
\end{lstlisting}

%~~~~~Aufgabe 3
\section*{Aufgabe 3}
\begin{lstlisting}
%%hier dein code
\end{lstlisting}


\end{document}

\documentclass[12pt,a4paper,oneside,ngerman]{article} 
\usepackage[left=3cm,right=3cm,top=2.5cm]{geometry} % Groesse der Seitenraender definieren
\usepackage[utf8]{inputenc} % utf8 encoding
\usepackage{hyperref}
\usepackage{ngerman}[babel]
\usepackage{graphicx}
\usepackage{tikz} % Automaten, Graphen, ... zeichnen
\usepackage{tikz-qtree} % Paket fuer Tikz Graphen-Baeume
\usetikzlibrary{arrows,shapes,automata} % Bestimmte tikz-Befehle benutzen
\usepackage{amsmath,amssymb} % Mathe-Formeln und -Ausdruecke
\usepackage{listings} % Code-Ausschnitte einbinden
\usepackage{xcolor} % Eigene Farben definieren
\usepackage{colortbl} % Farben verwenden in Tabellen
\usepackage{wrapfig} % Bilder von Text umfliessen lassen
\usepackage{multicol} % Mehrspaltigen Text schreiben
\usepackage{stmaryrd} % Fuer Symbole wie zu Beispiel Widerspruchspfeil
\usepackage{caption}
\usepackage{totpages}

\usepackage{circuitikz}
\usepackage{amsmath}
\usepackage{graphicx}
\usepackage{subfigure}
\usepackage{tikz}
\usepackage{float}

\lstset{language=Java,
basicstyle=\small \ttfamily,
keywordstyle=\color{javapurple}\bfseries,
stringstyle=\color{javared},
commentstyle=\color{javagreen},
morecomment=[s][\color{javadocblue}]{/**}{*/},
numbers=left,
numberstyle=\tiny\color{black},
stepnumber=1,
numbersep=10pt,
tabsize=1,
showspaces=false,
showstringspaces=false,
breaklines=true}
\lstset{literate=%
  {Ö}{{\"O}}1
  {Ä}{{\"A}}1
  {Ü}{{\"U}}1
  {ß}{{\ss}}1
  {ü}{{\"u}}1
  {ä}{{\"a}}1
  {ö}{{\"o}}1
}

% Beliebige RGB Farben definieren:
\definecolor{gold}{rgb}{0.83, 0.69, 0.15}
\definecolor{magenta}{rgb}{0.79, 0.08, 0.48}
\definecolor{javared}{rgb}{0.6,0,0} % for strings
\definecolor{javagreen}{rgb}{0.25,0.5,0.35} % comments
\definecolor{javapurple}{rgb}{0.5,0,0.35} % keywords
\definecolor{javadocblue}{rgb}{0.25,0.35,0.75} % javadoc

% Beliebige RGB Farben definieren:
\definecolor{gold}{rgb}{0.83, 0.69, 0.15}
\definecolor{magenta}{rgb}{0.79, 0.08, 0.48}

% Titel in Kopfzeilen
\usepackage{fancyhdr}
\pagestyle{fancy}
\fancyhf{}
\setlength{\headheight}{20pt}

% Seitenumbrueche werden nicht mehr eingerueckt
\setlength{\parindent}{0em}


% % % % % % % % % % % % % % % % % % % % % % % % % % % % % % 
%Variablen
% % % % % % % % % % % % % % % % % % % % % % % % % % % % % % 
\newcommand{\fach}{Objektorientierte Modellierung und Programmierung}
\newcommand{\dokumentenTitel}{Abgabe Uebungsblatt Nr.08}
\newcommand{\Abgabe}{16.06.2020, 12:00 Uhr}
\newcommand{\memberOne}{Marius Birk}
\newcommand{\memberTwo}{Pieter Vogt}


\newcommand{\tutor}{ Florian Brandt }
% % % % % % % % % % % % % % % % % % % % % % % % % % % % % 

% Kopfzeile auf jeder Seite:
\fancyhead[R]{\dokumentenTitel} % Dokument-Titel
\fancyhead[C]{}
\fancyhead[L]{\memberOne, \memberTwo} % Autorennamen
\renewcommand{\headrulewidth}{0.4pt} %obere Trennlinie

% Fußzeile auf jeder Seite:
\fancyfoot[C]{Seite \thepage \ von \ref{TotPages}} %Seitennummer
\renewcommand{\footrulewidth}{0.4pt} %untere Trennlinie

% Nun beginnt das eigentliche Dokument
\begin{document}
	\thispagestyle{plain} % Keine Kopfzeile auf erster Seite, aber Seitenzahl wird angezeigt
	
	\begin{multicols}{2} % Beginnt zweispaltigen Text fuer Header auf erster Seite
		\hspace{-1cm} % Linken Header-Teil 1cm nach links schieben.
		% Tabelle fuer linke Seite vom Header der ersten Seite
		\begin{tabular}{ll} % Mit l werden die Eintraege linksbuendig
			Autoren: & \memberOne \\ % Zwischen jeder Spalte ein & einfuegen
			& \memberTwo \\
% beendet eine Tabellenzeile 
			Tutor: & \tutor \\  
		\end{tabular}
		
		\columnbreak % Nun beginnt die rechte Seite des Headers
		\hspace{-1cm} % Rechten Header-Teil 1cm nach links schieben.
		% Tabelle fuer rechte Seite vom Header der ersten Seite
		\begin{tabular}{ll} % p{1cm} bewirkt, dass die rechte Spalte 6cm breit ist.
			Abgabe: & \Abgabe \\ % Zwischen jeder Spalte ein & einfuege
			Smileys: &  
			%Mit diesem Befehl wird die Zeilenhoehe der folgenden Tabelle um 20% erhoeht.   
			\renewcommand{\arraystretch}{1.2} 
			% Nun kommt eine innere Tabelle in der aeusseren Tabelle, mit der eine Punktetabelle fuer den Tutor erstellt wird:  
			\begin{tabular}{|p{0.8cm}|p{0.8cm}|p{0.8cm}|p{0.8cm}|p{0.8cm}|}
				\hline A1 & A2 & A3 &$\sum\limits^{ }$ \\ \hline
				& & & \\ \hline    
			\end{tabular} \\
		\end{tabular}
		
	\end{multicols} % Beendet zweispaltigen Text
	
	\begin{center}
		\Large{\fach} \\
		\LARGE{\dokumentenTitel} \\
		\small
		$($Alle allgemeinen Definitionen aus der Vorlesung haben in diesem Dokument bestand, es sei den sie erhalten eine explizit andere Definition.$)$
	\end{center}


%~~~~~DOKUMENT ANFANG~~~~~%

%~~~~~Aufgabe 1
\section*{Aufgabe 1 / Aufgabe 2}
\begin{lstlisting}
	import java.io.*;
	import java.nio.charset.Charset;
	import java.util.ArrayList;
	import java.util.List;
	
	public class Lecture {
		private String number = "";
		private String title = "";
		private String shortTitle = "";
		private String semester = "";
		private List<Lecturer> lecturers = new ArrayList<>();
		private List<Date> schedule = new ArrayList<>();
	
		public Lecture(String number, String title, String shortTitle, String semester) {
			super();
			this.number = number;
			this.title = title;
			this.shortTitle = shortTitle;
			this.semester = semester;
		}
	
		public String getNumber() {
			return number;
		}
	
		public void setNumber(String number) {
			this.number = number;
		}
	
		public String getTitle() {
			return title;
		}
	
		public void setTitle(String title) {
			this.title = title;
		}
	
		public String getShortTitle() {
			return shortTitle;
		}
	
		public void setShortTitle(String shortTitle) {
			this.shortTitle = shortTitle;
		}
	
		public String getSemester() {
			return semester;
		}
	
		public void setSemester(String semester) {
			this.semester = semester;
		}
	
		public List<Lecturer> getLecturers() {
			return lecturers;
		}
	
		public List<Date> getSchedule() {
			return schedule;
		}
		
		@Override
		public String toString() {
			StringBuilder result = new StringBuilder();
			result.append(number);
			result.append(": ");
			result.append(title);
			result.append(" (");
			result.append(shortTitle);
			result.append("), ");
			result.append(semester);
			result.append("\n\t");
			for (int i = 0; i < lecturers.size(); i++) {
				if (i > 0) {
					result.append(", ");
				}
				result.append(lecturers.get(i));
			}
			for (Date date : schedule) {
				result.append("\n\t- ");
				result.append(date);
			}
			result.append("\n");
			return result.toString();
		}
		public static Lecture load(String filename) throws IOException {
			Lecture result = null;
			InputStream in = null;
			try {
				in = new FileInputStream(filename);
				result = load(in);
			} finally {
				if (in != null) {
					in.close();
				}
			}
			return result;
		}
		public static Lecture load(InputStream in) throws IOException {
			String number;
			String title;
			String shortTitle;
			String semester;
			StringBuilder builder = new StringBuilder();
	
			int ch;
			int point = 33;
			boolean bool=false;
			String compare="";
			int[] replace = new int[]{1,2,3,4,5,6,7,8,9,10,11,12,13,14,15,16,17,18,19,20,21,22,23,24,25,26,27,28,29,30,31,32};
			try{
				while((ch = in.read()) != -1){
					if(ch==0){
						continue;
					}else{
						for(int i = 0; i<=replace.length-1;i++){
							if(ch==replace[i]){
								ch=33;
								bool=true;
							}else{
								bool=false;
							}
						}
					}
					builder.append((char) ch);
				}
			} catch (IOException e) {
				e.printStackTrace();
			}
			String[] split = builder.toString().split("!");
			number =split[1];
			title = split[2]+" "+split[3]+" "+split[4]+" "+split[5]+" "+split[6];
			shortTitle= split[7];
			semester= split[8];
	
			Lecture lect = new Lecture(number, title, shortTitle, semester);
	
			Lecturer snape = new Lecturer(split[10],split[11]);
					lect.lecturers.add(snape);
			Lecturer umbridge = new Lecturer(split[12], split[13]);
					lect.lecturers.add(umbridge);
			Lecturer lupin= new Lecturer(split[14], split[15]);
					lect.lecturers.add(lupin);
			return lect;
		}
		
		public static void saveText(String filename, Lecture data) throws IOException {
			PrintWriter out = null;
			try{
				out = new PrintWriter(new FileOutputStream(filename));
				out.print(data.getNumber()+"\n"+data.getTitle()+"\n"+data.getShortTitle()+"\n"+data.getSemester()+"\n"+data.getLecturers()+"\n"+data.getSchedule());
			}catch(FileNotFoundException e){
				e.printStackTrace();
			}finally {
				out.close();
			}
		}
	
		public static Lecture loadText(String filename) throws IOException {
			BufferedReader in = null;
			Lecture lect= null;
			try{
				in=new BufferedReader(new FileReader(filename));
				String zeile = null;
				ArrayList<String> tmp = new ArrayList<>();
				while((zeile=in.readLine())!=null){
					tmp.add(zeile);
				}
				lect = new Lecture(tmp.get(0), tmp.get(1), tmp.get(2), tmp.get(3));
				String[] tmp2 = tmp.get(4).split(",");
				tmp2[0]=tmp2[0].substring(1);
				tmp2[tmp2.length-1]=tmp2[tmp2.length-1].substring(0, tmp2[tmp2.length-1].length()-1);
	
				for(int i = 0; i<=tmp2.length-1; i++){
					String[] tmp3= tmp2[i].split(" ");
					if(tmp3.length>2) {
						lect.lecturers.add(new Lecturer(tmp3[tmp3.length-2], tmp3[tmp3.length-1]));
					}else{
						lect.lecturers.add(new Lecturer(tmp3[0], tmp3[1]));
					}
				}
			}finally {
				in.close();
			}
			return lect;
		}
	}	
\end{lstlisting}

\section*{Aufgabe 3}
\subsection*{Klasse A3}
\begin{lstlisting}
	import java.io.BufferedReader;
	import java.io.IOException;
	import java.net.URL;
	import java.net.URLConnection;
	
	public class A3 {
		public static void main(String[] args) throws IOException {
			String url = "https://uol.de/en/computingscience/se/publications";
			Connect connect = new Connect(url);
			BufferedReader in = null;
			in = connect.connect();
			connect.count(in);
		}
	}
\end{lstlisting}
\\ 
\subsection*{Klasse Connect}
\begin{lstlisting}
import java.io.*;
import java.net.URL;
import java.net.URLConnection;
import java.util.ArrayList;

public class Connect {
    private String url;
    private int proceeding;
    private int articel;
    private int phdthesis;
    private String input;
    public Connect(String url) {
        this.url =url;
        proceeding =0;
        articel = 0;
        phdthesis = 0;
    }
    public BufferedReader connect() throws IOException {
        URL web = new URL(url);
        BufferedReader in=null;
        StringBuilder inputLine = new StringBuilder();
        try{
            URLConnection connect = web.openConnection();
            in = new BufferedReader(new InputStreamReader(connect.getInputStream()));

        }catch(IOException e){
            e.printStackTrace();
        }finally{
            return in;
        }

    }
    public void count(BufferedReader in) throws IOException {
        ArrayList<String> tmp = new ArrayList<>();
        while((input=in.readLine())!= null){
            tmp.add(input);
        }
        for(int i = 0; i<tmp.size();i++){
            if(tmp.get(i).contains("inproceedings")){
                proceeding+=1;
            }
            if(tmp.get(i).contains("[article]")){
                articel+=1;
            }
            if(tmp.get(i).contains("[phdthesis]")){
                phdthesis+=1;
            }
        }
        System.out.print("proceeding: "+proceeding+", article: "+articel+", phdthesis: "+phdthesis);
    }
}

\end{lstlisting}
\end{document}

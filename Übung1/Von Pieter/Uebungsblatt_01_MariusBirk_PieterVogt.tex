\documentclass[12pt,a4paper,oneside,ngerman]{article} 
\usepackage[left=3cm,right=3cm,top=2.5cm]{geometry} % Groesse der Seitenraender definieren
\usepackage[utf8]{inputenc} % utf8 encoding
\usepackage{hyperref}
\usepackage{ngerman}
\usepackage{enumerate}
\usepackage{amsmath,amssymb} % Mathe-Formeln und -Ausdruecke
\usepackage{listings} % Code-Ausschnitte einbinden
\usepackage{xcolor} % Eigene Farben definieren
\usepackage{colortbl} % Farben verwenden in Tabellen
\usepackage{wrapfig} % Bilder von Text umfliessen lassen
\usepackage{multicol} % Mehrspaltigen Text schreiben
\usepackage{caption}
\usepackage{graphicx}
\usepackage{amsmath}
\usepackage{float}

\lstset{language=Java,
basicstyle=\small \ttfamily,
keywordstyle=\color{javapurple}\bfseries,
stringstyle=\color{javared},
commentstyle=\color{javagreen},
morecomment=[s][\color{javadocblue}]{/**}{*/},
numbers=left,
numberstyle=\tiny\color{black},
stepnumber=1,
numbersep=10pt,
tabsize=1,
showspaces=false,
showstringspaces=false,
breaklines}

% Beliebige RGB Farben definieren:
\definecolor{gold}{rgb}{0.83, 0.69, 0.15}
\definecolor{magenta}{rgb}{0.79, 0.08, 0.48}
\definecolor{javared}{rgb}{0.6,0,0} % for strings
\definecolor{javagreen}{rgb}{0.25,0.5,0.35} % comments
\definecolor{javapurple}{rgb}{0.5,0,0.35} % keywords
\definecolor{javadocblue}{rgb}{0.25,0.35,0.75} % javadoc

% Titel in Kopfzeilen
\usepackage{fancyhdr}
\pagestyle{fancy}
\setlength{\headheight}{20pt}

% Seitenumbrueche werden nicht mehr eingerueckt
\setlength{\parindent}{0em}
\setlength{\parskip}{0.25em}


% % % % % % % % % % % % % % % % % % % % % % % % % % % % % % 
%Variablen/Befehle -> Mit euren Informationen füllen!
% % % % % % % % % % % % % % % % % % % % % % % % % % % % % % 
\newcommand{\fach}{Objektorientierte Modellierung und Programmierung}
\newcommand{\dokumentenTitel}{Übung 01}
\newcommand{\tutorium}{B/G}
\newcommand{\memberOne}{Marius Birk}
\newcommand{\memberTwo}{Pieter Vogt}
\newcommand{\group} {Ladies Night}
% % % % % % % % % % % % % % % % % % % % % % % % % % % % % 

% Kopfzeile auf jeder Seite:
\fancyhead[R]{\dokumentenTitel, \group} % Dokument-Titel
\fancyhead[L]{\memberOne, \memberTwo,} % Autorennamen

% % % % % % % % % % % % % % % % % % % % % % % % % % % % % % 
% Hier ist die Kopfzeile und die ganzen Formalia
% % % % % % % % % % % % % % % % % % % % % % % % % % % % % %

\begin{document}
	\thispagestyle{plain} % Keine Kopfzeile auf erster Seite, aber Seitenzahl wird angezeigt
	
	\begin{multicols}{2} % Beginnt zweispaltigen Text fuer Header auf erster Seite
		\hspace{-1cm} % Linken Header-Teil 1cm nach links schieben.
		% Tabelle fuer linke Seite vom Header der ersten Seite
		\begin{tabular}{ll} % Mit l werden die Eintraege linksbuendig
			Gruppe: & Ladies Night \\
			Autoren: & Marius Birk \\ % Zwischen jeder Spalte ein & einfuegen
			& Pieter Vogt \\
		\end{tabular}
		
		\columnbreak % Nun beginnt die rechte Seite des Headers
		\hspace{-1cm} % Rechten Header-Teil 1cm nach links schieben.
		% Tabelle fuer rechte Seite vom Header der ersten Seite
		\raggedleft \begin{tabular}{ll}
			Tutorium: &  Gruppe B/G \\
			Punkte: &     
			\renewcommand{\arraystretch}{1.2} %Mit diesem Befehl wird die Zeilenhoehe der folgenden Tabelle um 20% erhoeht.
			% Nun kommt eine innere Tabelle in der aeusseren Tabelle, mit der eine Punktetabelle fuer den Tutor erstellt wird:  
			
% % % % % % % % % % % % % % % % % % % % % % % % % % % % % %
% Punktetabelle: Anpassen je nach Aufgabenanzahl :)
% % % % % % % % % % % % % % % % % % % % % % % % % % % % % %
			\begin{tabular}{|p{0.8cm}|p{0.8cm}|} %Spaltenanzahl und breite
				\hline A1&$\sum\limits^{ }$ \\ \hline %obere Zeile
				& \\ \hline   %untere Zeile
			\end{tabular}
		\end{tabular}	
	\end{multicols} % Beendet zweispaltigen Text
	
% % % % % % % % % % % % % % % % % % % % % % % % % % % % % % 
% Nun beginnt das eigentliche Dokument:
% % % % % % % % % % % % % % % % % % % % % % % % % % % % % %
	\begin{center}
		\Large{\fach} \\
		\LARGE{\dokumentenTitel} \\
		\small
\end{center}
ANMERKUNG: Marius Birk(Tutorium B) und Pieter Vogt (Tutorium G) arbeiten zusammen und geben gemeinsam den bearbeiteten Zettel ab.
\section*{Aufgabe 1}
\subsection*{UML-Architektur}
\includegraphics[width=\linewidth]{Uebung1}
\subsection*{Implementierung}
Der Code für die Klasse 'Animal' sieht aus wie folgt:
\begin{lstlisting}
public class Animal {
   //FIELDS
   private String name;
   private Animal[] eatsAnimals = new Animal[10];
   private Plant[] eatsPlants = new Plant[10];
   //CONSTRUCTOR
   public Animal(String name) {
      this.name = name;
   }
   //GETTER SETTER
   public void setName(String name) {
      this.name = name;
   }
   public String getName() {
      return name;
   }
   public void setPlantDiet(Plant plant) {
      for (int i = 0; i < eatsPlants.length; i++) {
         if (eatsPlants[i] == null) {
            eatsPlants[i] = plant;
            return;
         }
         return;
      }
   }
   public void setAnimalDiet(Animal animal) {
      for (int i = 0; i < eatsAnimals.length; i++) {
         if (eatsAnimals[i] == null) {
            eatsAnimals[i] = animal;
            return;
         }
         return;
      }
   }
   //PUBLIC METHODS
   public void getDiet() {
      if (isCarnivore()) {
         System.out.println(this.name + " is a carnivore");
      } else if (isHerbivore()) {
         System.out.println(this.name + " is a herbivore");
      } else System.out.println(this.name + " is a omnivore");
   }
   //HELPER METHODS

   private Boolean isCarnivore() {
      if (!likesPlant()) {
         return true;
      } else return false;
   
   private Boolean isHerbivore() {
      if (!likesMeat()) {
         return true;
      } else return false;
   }
   private Boolean isOmnivore() {
      if (likesMeat() && likesPlant()) {
         return true;
      } else return false;
   }
   private boolean likesMeat() {
      for (int i = 0; i < eatsAnimals.length; i++) {
         if (eatsAnimals[i] != null) {
            return true;
         }
      }
      return false;
   }
   private boolean likesPlant() {
      for (int i = 0; i < eatsPlants.length; i++) {
         if (eatsPlants[i] != null) {
            return true;
         }
      }
      return false;
   }
}
\end{lstlisting}
Der Code für die Klasse 'Plant' sieht aus wie folgt:
\begin{lstlisting}
public class Plant {
   //FIELDS
   private String name;
   private String description;
   //GETTER SETTER
   public String getName() {
      return name;
   }
   public String getDescription() {
      return description;
   }
   public void setName(String name) {
      this.name = name;
   }
   public void setDescription(String description) {
      this.description = description;
   }
}
\end{lstlisting}
Der Code für das Programm 'Biotest' sieht aus wie folgt:

Die UML-Architektur der Objektebene nach ausführung von 'Biotest' sieht wie folgt aus:


\end{document}

\documentclass[12pt,a4paper,oneside,ngerman]{article} 
\usepackage[left=3cm,right=3cm,top=2.5cm]{geometry} % Groesse der Seitenraender definieren
\usepackage[utf8]{inputenc} % utf8 encoding
\usepackage{hyperref}
\usepackage{ngerman}[babel]
\usepackage{graphicx}
\usepackage{tikz} % Automaten, Graphen, ... zeichnen
\usepackage{tikz-qtree} % Paket fuer Tikz Graphen-Baeume
\usetikzlibrary{arrows,shapes,automata} % Bestimmte tikz-Befehle benutzen
\usepackage{amsmath,amssymb} % Mathe-Formeln und -Ausdruecke
\usepackage{listings} % Code-Ausschnitte einbinden
\usepackage{xcolor} % Eigene Farben definieren
\usepackage{colortbl} % Farben verwenden in Tabellen
\usepackage{wrapfig} % Bilder von Text umfliessen lassen
\usepackage{multicol} % Mehrspaltigen Text schreiben
\usepackage{stmaryrd} % Fuer Symbole wie zu Beispiel Widerspruchspfeil
\usepackage{caption}
\usepackage{totpages}

\usepackage{circuitikz}
\usepackage{amsmath}
\usepackage{graphicx}
\usepackage{subfigure}
\usepackage{tikz}
\usepackage{float}

\lstset{language=Java,
basicstyle=\small \ttfamily,
keywordstyle=\color{javapurple}\bfseries,
stringstyle=\color{javared},
commentstyle=\color{javagreen},
morecomment=[s][\color{javadocblue}]{/**}{*/},
numbers=left,
numberstyle=\tiny\color{black},
stepnumber=1,
numbersep=10pt,
tabsize=1,
showspaces=false,
showstringspaces=false,
breaklines=true}
\lstset{literate=%
  {Ö}{{\"O}}1
  {Ä}{{\"A}}1
  {Ü}{{\"U}}1
  {ß}{{\ss}}1
  {ü}{{\"u}}1
  {ä}{{\"a}}1
  {ö}{{\"o}}1
}

% Beliebige RGB Farben definieren:
\definecolor{gold}{rgb}{0.83, 0.69, 0.15}
\definecolor{magenta}{rgb}{0.79, 0.08, 0.48}
\definecolor{javared}{rgb}{0.6,0,0} % for strings
\definecolor{javagreen}{rgb}{0.25,0.5,0.35} % comments
\definecolor{javapurple}{rgb}{0.5,0,0.35} % keywords
\definecolor{javadocblue}{rgb}{0.25,0.35,0.75} % javadoc

% Beliebige RGB Farben definieren:
\definecolor{gold}{rgb}{0.83, 0.69, 0.15}
\definecolor{magenta}{rgb}{0.79, 0.08, 0.48}

% Titel in Kopfzeilen
\usepackage{fancyhdr}
\pagestyle{fancy}
\fancyhf{}
\setlength{\headheight}{20pt}

% Seitenumbrueche werden nicht mehr eingerueckt
\setlength{\parindent}{0em}


% % % % % % % % % % % % % % % % % % % % % % % % % % % % % % 
%Variablen
% % % % % % % % % % % % % % % % % % % % % % % % % % % % % % 
\newcommand{\fach}{Objektorientierte Modellierung und Programmierung}
\newcommand{\dokumentenTitel}{Abgabe Uebungsblatt Nr.05}
\newcommand{\Abgabe}{26.05.2020, 12:00 Uhr}
\newcommand{\memberOne}{Marius Birk}
\newcommand{\memberTwo}{Pieter Vogt}


\newcommand{\tutor}{ Florian Brandt }
% % % % % % % % % % % % % % % % % % % % % % % % % % % % % 

% Kopfzeile auf jeder Seite:
\fancyhead[R]{\dokumentenTitel} % Dokument-Titel
\fancyhead[C]{}
\fancyhead[L]{\memberOne, \memberTwo} % Autorennamen
\renewcommand{\headrulewidth}{0.4pt} %obere Trennlinie

% Fußzeile auf jeder Seite:
\fancyfoot[C]{Seite \thepage \ von \ref{TotPages}} %Seitennummer
\renewcommand{\footrulewidth}{0.4pt} %untere Trennlinie

% Nun beginnt das eigentliche Dokument
\begin{document}
	\thispagestyle{plain} % Keine Kopfzeile auf erster Seite, aber Seitenzahl wird angezeigt
	
	\begin{multicols}{2} % Beginnt zweispaltigen Text fuer Header auf erster Seite
		\hspace{-1cm} % Linken Header-Teil 1cm nach links schieben.
		% Tabelle fuer linke Seite vom Header der ersten Seite
		\begin{tabular}{ll} % Mit l werden die Eintraege linksbuendig
			Autoren: & \memberOne \\ % Zwischen jeder Spalte ein & einfuegen
			& \memberTwo \\
% beendet eine Tabellenzeile 
			Tutor: & \tutor \\  
		\end{tabular}
		
		\columnbreak % Nun beginnt die rechte Seite des Headers
		\hspace{-1cm} % Rechten Header-Teil 1cm nach links schieben.
		% Tabelle fuer rechte Seite vom Header der ersten Seite
		\begin{tabular}{ll} % p{1cm} bewirkt, dass die rechte Spalte 6cm breit ist.
			Abgabe: & \Abgabe \\ % Zwischen jeder Spalte ein & einfuege
			Smileys: &  
			%Mit diesem Befehl wird die Zeilenhoehe der folgenden Tabelle um 20% erhoeht.   
			\renewcommand{\arraystretch}{1.2} 
			% Nun kommt eine innere Tabelle in der aeusseren Tabelle, mit der eine Punktetabelle fuer den Tutor erstellt wird:  
			\begin{tabular}{|p{0.8cm}|p{0.8cm}|p{0.8cm}|p{0.8cm}|p{0.8cm}|}
				\hline A1 & A2 & A3 &$\sum\limits^{ }$ \\ \hline
				& & & \\ \hline    
			\end{tabular} \\
		\end{tabular}
		
	\end{multicols} % Beendet zweispaltigen Text
	
	\begin{center}
		\Large{\fach} \\
		\LARGE{\dokumentenTitel} \\
		\small
		$($Alle allgemeinen Definitionen aus der Vorlesung haben in diesem Dokument bestand, es sei den sie erhalten eine explizit andere Definition.$)$
	\end{center}


%~~~~~DOKUMENT ANFANG~~~~~%

%~~~~~Aufgabe 1
\section*{Aufgabe 1}
\begin{lstlisting}
   import java.lang.reflect.Method;

   class Util {
   
      // liefert die kleinste Zahl des uebergebenen Arrays
      public static int minimum(int[] values) throws ArrayNull {
         int min =0;
         if(values == null){
            throw new ArrayNull();
         }else {
             min = values[0];
            for (int i = 1; i < values.length; i++) {
               if (values[i] < min) {
                  min = values[i];
               }
            }
         }
         return min;
      }
      // konvertiert den uebergebenen String in einen int-Wert
      public static int toInt(String str) throws emptyString {
         int result = 0, factor = 1;
         if(str.equals(" ")){
            throw new emptyString();
         }else {
            char ch = str.charAt(0);
            switch (ch) {
               case '-':
                  factor = -1;
                  break;
               case '+':
                  factor = 1;
                  break;
               default:
                  result = ch - '0';
            }
            for (int i = 1; i < str.length(); i++) {
               ch = str.charAt(i);
               int ziffer = ch - '0';
               result = result * 10 + ziffer;
            }
         }
         return factor * result;
      }
   
      // liefert die Potenz von zahl mit exp,
      // also zahl "hoch" exp (number to the power of exp)
      public static long power(long number, int exp) {
         if (exp == 0) {
            return 1L;
         }
         return number * Util.power(number, exp - 1);
      }
   }
   
   public class UtilTest {
      // Testprogramm
      public static void main(String[] args) throws ClassNotFoundException, NoSuchMethodException {
         String eingabe="";
         try{
            Class IO = Class.forName("IO");
            ClassLoader classLoader = IO.getClassLoader();
            Class IO2 = Class.forName("IO", true, classLoader);
   
            for(Method readString: IO.getDeclaredMethods()){
               if(readString.getName().equals("readString")){
                  //eingabe = IO.readString("Zahl: ");
               }
            }
         }catch(ClassNotThere e){
            e.printStackTrace();
            System.out.print(e);
         }
         int zahl = Util.toInt(eingabe);
         System.out.println(zahl + " hoch " + zahl + " = " + Util.power(zahl, zahl));
         System.out.println(Util.minimum(new int[] { 1, 6, 4, 7, -3, 2 }));
         System.out.println(Util.minimum(new int[0]));
         System.out.println(Util.minimum(null));
      }
   }
   
   class ArrayNull extends RuntimeException {
      public ArrayNull() {
         super("Das Array ist leer");
      }
   
      public ArrayNull(String fehlermeldung) {
         super(fehlermeldung);
      }
   }
   class emptyString extends RuntimeException {
      public emptyString() {
         super("Der String ist leer");
      }
   
      public emptyString(String fehlermeldung) {
         super(fehlermeldung);
      }
   }
   class ClassNotThere extends ClassNotFoundException {
      public ClassNotThere() {
         super("Klasse existiert nicht.");
      }
   
      public ClassNotThere(String fehlermeldung) {
         super(fehlermeldung);
      }
   }
   
\end{lstlisting}

%~~~~~Aufgabe 2
\section*{Aufgabe 2}
\subsection*{Bill.java}
\begin{lstlisting}
import java.util.ArrayList;

public class Bill {

   //fields

   String name;
   double billPrice = 0;
   ArrayList<BillItem> items = new ArrayList<>();

   //methods

   public void add(CarPart part) {
      items.add(new BillItem(part));
   }

   //getter - setter

   public double getTotalPrice() {
      return billPrice;
   }

   public String toString() {
      StringBuffer tempString = new StringBuffer("Receipt for Bill: ");
      double receipTotal = 0;
      tempString.append(this.name);
      tempString.append("\n");
      for (int i = 0; i < items.size(); i++) {
         tempString.append(items.get(i).item.getName());//add ItemName
         tempString.append("\t");
         tempString.append(items.get(i).item.getPrice());//add ItemPrice
         tempString.append("\n");
         receipTotal = receipTotal + items.get(i).item.getPrice();
      }
      tempString.append("\n");
      Math.nextUp(receipTotal);//doesn't work for some reason.
      tempString.append("In Total this receipt is: " + receipTotal);
      String output = tempString.toString();
      return output;
   }

   //constructors

   public Bill(String name) {
      this.name = name;
   }

   //nested classes

   private class BillItem {

      //fields

      CarPart item;

      //methods

      //getter - setter

      public CarPart getItem() {
         return item;
      }

      public void setItem(CarPart item) {
         this.item = item;
      }

      public BillItem(CarPart item) {
         this.item = item;
      }
   }

}

\end{lstlisting}

\subsection*{Car.java}
\begin{lstlisting}
import java.util.ArrayList;

public class Car {
   ArrayList<CarPart> parts = new ArrayList<>();
}

\end{lstlisting}

\subsection*{CarComponent.java}
\begin{lstlisting}
public interface CarComponent {
   public String getName();
}

\end{lstlisting}

\subsection*{CarPart.java}
\begin{lstlisting}
public class CarPart implements CarComponent {
   String name;
   double price;

   @Override
   public String getName() {
      return null;
   }

   public double getPrice() {
      return price;
   }

   public static class Seat extends CarPart {
      String name = new String("Seat");
      double price = 2000.0;

      @Override
      public String getName() {
         return name;
      }

      public double getPrice() {
         return price;
      }
   }

   public static class Wheel extends CarPart {
      String name = new String("Wheel");

      double price = 1000.0;

      @Override
      public String getName() {
         return name;
      }

      public double getPrice() {
         return price;
      }
   }

   public static class Motor extends CarPart {
      String name = new String("Motor");

      double price = 100000;

      @Override
      public String getName() {
         return name;
      }

      public double getPrice() {
         return price;
      }
   }
}

\end{lstlisting}
\newpage
\subsection*{Main.java}
\begin{lstlisting}
public class Main {
   public static void main(String[] args) {
      Bill bill = new Bill("Rolls Royce");
      bill.add(new CarPart.Motor());
      bill.add(new CarPart.Seat());
      bill.add(new CarPart.Wheel());
      bill.add(new CarPart.Wheel());
      bill.add(new CarPart.Wheel());
      bill.add(new CarPart.Wheel());
      System.out.println(bill.toString());
   }
}
\end{lstlisting}

%~~~~~Aufgabe 3
\section*{Aufgabe 3}
\subsection{Class Person}
\begin{lstlisting}
   import java.util.ArrayList;

   class Person<T> implements Older<T>{
      private String name;
      private int age;
       private Object other;
  
  
       public Person( String name,  int i) {
          this.name = name;
          this.age = i;
      }
  
      public String getName() {
          return name;
      }
  
      public int getAge(){
          return age;
      }
  
      @Override public <T extends Person> boolean isOlder(T other) {
          if(this.getAge() > other.getAge()){
              return true;
          }else{
              return false;
          }
      }
  }
  
  interface Older<T > {
      public <T extends Person> boolean isOlder(T other);
  }
  
  class Group<T extends Person> {
      public Group() {
  
      }
      ArrayList<T> group = new ArrayList<>();
  
      public void add(Person member) {
                  group.add((T) member);
      }
      public T getOldest(){
          int index=0;
          for(int i = 0; i<=group.size()-1; i++){
              try{
                  if(group.get(i+1)!=null){
                      if(group.get(i).getAge()>group.get(i+1).getAge() ){
                          index=i;
                      }else{
                          index=+1;
                      }
                  }
              }catch(Exception e){
                  if(group.get(i).getAge() > group.get(index).getAge()){
                      index=index+1;
                  }
              }
          }
         return group.get(index);
      }
  
      public String get(int i) {
          return group.get(i).getName();
      }
  }
\end{lstlisting}
\subsection{Class TestGroup}
\begin{lstlisting}
   public class TestGroup {
      public static void main(String[] args) {
         Group<Person> group = new Group<>();
         group.add(new Person("Alice", 25));
         group.add(new Person("Bob", 23));
         group.add(new Person("Carl", 26));
         System.out.println(group.getOldest().getName());
      }
   }
\end{lstlisting}

\end{document}
